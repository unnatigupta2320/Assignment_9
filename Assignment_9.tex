\documentclass[journal,12pt,twocolumn]{IEEEtran}

\usepackage{setspace}
\usepackage{gensymb}

\singlespacing


\usepackage[cmex10]{amsmath}

\usepackage{amsthm}

\usepackage{mathrsfs}
\usepackage{txfonts}
\usepackage{stfloats}
\usepackage{bm}
\usepackage{cite}
\usepackage{cases}
\usepackage{subfig}

\usepackage{longtable}
\usepackage{multirow}

\usepackage{enumitem}
\usepackage{mathtools}
\usepackage{steinmetz}
\usepackage{tikz}
\usepackage{circuitikz}
\usepackage{verbatim}
\usepackage{tfrupee}
\usepackage[breaklinks=true]{hyperref}
\usepackage{graphicx}
\usepackage{tkz-euclide}
\usepackage{float}

\usetikzlibrary{calc,math}
\usepackage{listings}
    \usepackage{color}                                            %%
    \usepackage{array}                                            %%
    \usepackage{longtable}                                        %%
    \usepackage{calc}                                             %%
    \usepackage{multirow}                                         %%
    \usepackage{hhline}                                           %%
    \usepackage{ifthen}                                           %%
    \usepackage{lscape}     
\usepackage{multicol}
\usepackage{chngcntr}

\DeclareMathOperator*{\Res}{Res}

\renewcommand\thesection{\arabic{section}}
\renewcommand\thesubsection{\thesection.\arabic{subsection}}
\renewcommand\thesubsubsection{\thesubsection.\arabic{subsubsection}}

\renewcommand\thesectiondis{\arabic{section}}
\renewcommand\thesubsectiondis{\thesectiondis.\arabic{subsection}}
\renewcommand\thesubsubsectiondis{\thesubsectiondis.\arabic{subsubsection}}


\hyphenation{op-tical net-works semi-conduc-tor}
\def\inputGnumericTable{}                                 %%

\lstset{
%language=C,
frame=single, 
breaklines=true,
columns=fullflexible
}
\begin{document}


\newtheorem{theorem}{Theorem}[section]
\newtheorem{problem}{Problem}
\newtheorem{proposition}{Proposition}[section]
\newtheorem{lemma}{Lemma}[section]
\newtheorem{corollary}[theorem]{Corollary}
\newtheorem{example}{Example}[section]
\newtheorem{definition}[problem]{Definition}

\newcommand{\BEQA}{\begin{eqnarray}}
\newcommand{\EEQA}{\end{eqnarray}}
\newcommand{\define}{\stackrel{\triangle}{=}}
\bibliographystyle{IEEEtran}
\providecommand{\mbf}{\mathbf}
\providecommand{\pr}[1]{\ensuremath{\Pr\left(#1\right)}}
\providecommand{\qfunc}[1]{\ensuremath{Q\left(#1\right)}}
\providecommand{\sbrak}[1]{\ensuremath{{}\left[#1\right]}}
\providecommand{\lsbrak}[1]{\ensuremath{{}\left[#1\right.}}
\providecommand{\rsbrak}[1]{\ensuremath{{}\left.#1\right]}}
\providecommand{\brak}[1]{\ensuremath{\left(#1\right)}}
\providecommand{\lbrak}[1]{\ensuremath{\left(#1\right.}}
\providecommand{\rbrak}[1]{\ensuremath{\left.#1\right)}}
\providecommand{\cbrak}[1]{\ensuremath{\left\{#1\right\}}}
\providecommand{\lcbrak}[1]{\ensuremath{\left\{#1\right.}}
\providecommand{\rcbrak}[1]{\ensuremath{\left.#1\right\}}}
\theoremstyle{remark}
\newtheorem{rem}{Remark}
\newcommand{\sgn}{\mathop{\mathrm{sgn}}}
\providecommand{\abs}[1]{\lvert#1\vert}
\providecommand{\res}[1]{\Res\displaylimits_{#1}} 
\providecommand{\norm}[1]{\lVert#1\rVert}
%\providecommand{\norm}[1]{\lVert#1\rVert}
\providecommand{\mtx}[1]{\mathbf{#1}}
\providecommand{\mean}[1]{E[ #1 ]}
\providecommand{\fourier}{\overset{\mathcal{F}}{ \rightleftharpoons}}
%\providecommand{\hilbert}{\overset{\mathcal{H}}{ \rightleftharpoons}}
\providecommand{\system}{\overset{\mathcal{H}}{ \longleftrightarrow}}
	%\newcommand{\solution}[2]{\textbf{Solution:}{#1}}
\newcommand{\solution}{\noindent \textbf{Solution: }}
\newcommand{\cosec}{\,\text{cosec}\,}
\providecommand{\dec}[2]{\ensuremath{\overset{#1}{\underset{#2}{\gtrless}}}}
\newcommand{\myvec}[1]{\ensuremath{\begin{pmatrix}#1\end{pmatrix}}}
\newcommand{\mydet}[1]{\ensuremath{\begin{vmatrix}#1\end{vmatrix}}}
\numberwithin{equation}{subsection}
\makeatletter
\@addtoreset{figure}{problem}
\makeatother
\let\StandardTheFigure\thefigure
\let\vec\mathbf
\renewcommand{\thefigure}{\theproblem}
\def\putbox#1#2#3{\makebox[0in][l]{\makebox[#1][l]{}\raisebox{\baselineskip}[0in][0in]{\raisebox{#2}[0in][0in]{#3}}}}
     \def\rightbox#1{\makebox[0in][r]{#1}}
     \def\centbox#1{\makebox[0in]{#1}}
     \def\topbox#1{\raisebox{-\baselineskip}[0in][0in]{#1}}
     \def\midbox#1{\raisebox{-0.5\baselineskip}[0in][0in]{#1}}
\vspace{3cm}
\title{ASSIGNMENT-9}
\author{Unnati Gupta}
\maketitle
\newpage
\bigskip
\renewcommand{\thefigure}{\theenumi}
\renewcommand{\thetable}{\theenumi}
Download all python codes from 
\begin{lstlisting}
https://github.com/unnatigupta2320/Assignment_9
\end{lstlisting}
%
and latex-tikz codes from 
%
\begin{lstlisting}
https://github.com/unnatigupta2320/Assignment_9
\end{lstlisting}
%
\section{Question No-2.63}
If $\vec{A}$ =$\myvec{1&2&3\\3&-2&1\\4&2&1}$, then show that \\$\vec{A}^3-23\vec{A}-40\vec{I}=0$.
\section{Solution}
Given that $\vec{A}$ =$\myvec{1&2&3\\3&-2&1\\4&2&1}$.
\begin{enumerate}
\item Calculating $\vec{A}^3$ :
\begin{itemize}
\item We will firstly calculate $\vec{A}^2$ :-
\begin{align}
\vec{A}^2&= \myvec{1&2&3\\3&-2&1\\4&2&1}\myvec{1&2&3\\3&-2&1\\4&2&1}
\\
&=\myvec{1+6+12&2-4+6&3+2+3\\3-6+4&6+4+2&9-2+1\\4+6+4&8-4+2&12+2+1}
\\
&\implies\vec{A}^2= \myvec{19&4&8\\1&12&8\\14&6&15}
\end{align}
\item Now, $\vec{A}^3$ can be given as :-
\begin{align}
 \vec{A}^3&=\vec{A}^2\vec{A}
 \\
 \vec{A}^3&=\myvec{19&4&8\\1&12&8\\14&6&15}\myvec{1&2&3\\3&-2&1\\4&2&1}
 \end{align}
 \begin{align}
 &=\myvec{19+12+32&38-8+16&57+4+8\\1+36+32&2-24+16&3+12+8\\14+18+60&28-12+30&42+6+15}
 \\
 &\implies \vec{A}^3=\myvec{63&46&69\\69&-6&23\\92&46&63}\label{eq:A3} 
\end{align}
\end{itemize}
\item Calculating 23$\vec{A}$ :-
\begin{align}
 23\vec{A}&=23\myvec{1&2&3\\3&-2&1\\4&2&1}
 \\
\implies 23\vec{A}&=\myvec{23&46&69\\69&-46&23\\92&46&23}\label{eq:23A}
\end{align}
\item Calculating 40$\vec{I}$ :-
\begin{align}
 40\vec{I}&=40\myvec{1&0&0\\0&1&0\\0&0&1}
 \\
 \implies 40\vec{I}&=\myvec{40&0&0\\0&40&0\\0&0&40}\label{eq:40I}
\end{align}
 \item Considering LHS of given equation we have:-
 \begin{align}
 LHS=\vec{A}^3-23\vec{A}-40\vec{I}   
 \end{align}
 \item Putting values from \eqref{eq:A3}, \eqref{eq:23A}
and \eqref{eq:40I} we get:-
\begin{multline}
\implies \text{LHS}=\myvec{63&46&69\\69&-6&23\\92&46&63}-
\myvec{23&46&69\\69&-46&23\\92&46&23}
\\-\myvec{40&0&0\\0&40&0\\0&0&40}
\end{multline}
\begin{align}
=\myvec{63-23-40&46-46&69-69\\69-69&-6+46-40&23-23\\92-92&46-46&63-23-40}
\end{align}
\begin{align}
&\implies \text{LHS}=\myvec{0&0&0\\0&0&0\\0&0&0}
\\
&\implies \text{LHS}=0=RHS
\\
&\implies \vec{A}^3-23\vec{A}-40\vec{I}=0
\end{align}
Hence, proved.
\end{enumerate}
\end{document}
